\title{Lock-Free Data Structures }

% for anonymous conference submission please enter your SUBMISSION ID
% instead of the author's name (and leave the affiliation blank) !!
\author[Kilian Probst]
{\parbox{\textwidth}{\centering Kilian Probst}
        \\
% For Computer Graphics Forum: Please use the abbreviation of your first name.
{\parbox{\textwidth}{\centering OTH Regensburg, Germany}}
}
% ------------------------------------------------------------------------

% if the Editors-in-Chief have given you the data, you may uncomment
% the following five lines and insert it here
%
\volume{2021}   % the volume in which the issue will be published;
\issue{1}     % the issue number of the publication
\pStartPage{1}      % set starting page


%-------------------------------------------------------------------------
\begin{document}

% uncomment for using teaser
% \teaser{
%  \includegraphics[width=\linewidth]{eg_new}
%  \centering
%   \caption{New EG Logo}
% \label{fig:teaser}
%}

\maketitle
%-------------------------------------------------------------------------
\begin{abstract}
   The ABSTRACT is to be in fully-justified italicized text,
   between two horizontal lines,
   in one-column format,
   below the author and affiliation information.
   Use the word ``Abstract'' as the title, in 9-point Times, boldface type,
   left-aligned to the text, initially capitalized.
   The abstract is to be in 9-point, single-spaced type.
   The abstract may be up to 3 inches (7.62 cm) long.
\end{abstract}
%-------------------------------------------------------------------------

\section{Introduction}

Please follow the steps outlined in this document very carefully when
submitting your manuscript to Eurographics.

You may as well use the \LaTeX\ source as a template to typeset your own
paper. In this case we encourage you to also read the \LaTeX\ comments
embedded in the document.

%-------------------------------------------------------------------------
\section{Instructions}

Please read the following carefully.

%-------------------------------------------------------------------------
\newpage


%-------------------------------------------------------------------------
\section{Conclusion}

Very much to conclude here.

%-------------------------------------------------------------------------
\newpage

\begin{figure*}[tbp]
  \centering
  \mbox{} \hfill
  % the following command controls the width of the embedded PS file
  % (relative to the width of the current column)
  \includegraphics[width=.3\linewidth]{oth.png}
  % replacing the above command with the one below will explicitly set
  % the bounding box of the PS figure to the rectangle (xl,yl),(xh,yh).
  % It will also prevent LaTeX from reading the PS file to determine
  % the bounding box (i.e., it will speed up the compilation process)
  % \includegraphics[width=.3\linewidth, bb=39 696 126 756]{sampleFig}
  \hfill
  \includegraphics[width=.3\linewidth]{oth.png}
  \hfill \mbox{}
  \caption{\label{fig:ex3}%
           For publications with color tables (i.e., publications not offering
           color throughout the paper) please \textbf{observe}:
           for the printed version -- and ONLY for the printed
           version -- color figures have to be placed in the last page.
           \newline
           For the electronic version, which will be converted to PDF before
           making it available electronically, the color images should be
           embedded within the document. Optionally, other multimedia
           material may be attached to the electronic version. }
\end{figure*}

\end{document}
