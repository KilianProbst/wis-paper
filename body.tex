\title[EG \LaTeX\ Author Guidelines]%
      {\LaTeX\ Author Guidelines for EUROGRAPHICS Proceedings Manuscripts}

% for anonymous conference submission please enter your SUBMISSION ID
% instead of the author's name (and leave the affiliation blank) !!
\author[D. Fellner \& S. Behnke]
{\parbox{\textwidth}{\centering D.\,W. Fellner\thanks{Chairman Eurographics Publications Board}$^{1,2}$
        and S. Behnke$^{2}$
%        S. Spencer$^2$\thanks{Chairman Siggraph Publications Board}
        }
        \\
% For Computer Graphics Forum: Please use the abbreviation of your first name.
{\parbox{\textwidth}{\centering $^1$TU Darmstadt \& Fraunhofer IGD, Germany\\
         $^2$Graz University of Technology, Institute of Computer Graphics and Knowledge Visualization, Austria
%        $^2$ Another Department to illustrate the use in papers from authors
%             with different affiliations
       }
}
}
% ------------------------------------------------------------------------

% if the Editors-in-Chief have given you the data, you may uncomment
% the following five lines and insert it here
%
\volume{2021}   % the volume in which the issue will be published;
\issue{1}     % the issue number of the publication
% \pStartPage{1}      % set starting page


%-------------------------------------------------------------------------
\begin{document}

% uncomment for using teaser
% \teaser{
%  \includegraphics[width=\linewidth]{eg_new}
%  \centering
%   \caption{New EG Logo}
% \label{fig:teaser}
%}

\maketitle
%-------------------------------------------------------------------------
\begin{abstract}
   The ABSTRACT is to be in fully-justified italicized text, 
   between two horizontal lines,
   in one-column format, 
   below the author and affiliation information. 
   Use the word ``Abstract'' as the title, in 9-point Times, boldface type, 
   left-aligned to the text, initially capitalized. 
   The abstract is to be in 9-point, single-spaced type.
   The abstract may be up to 3 inches (7.62 cm) long.
\end{abstract}  
%-------------------------------------------------------------------------

\section{Introduction}

Please follow the steps outlined in this document very carefully when
submitting your manuscript to Eurographics.

You may as well use the \LaTeX\ source as a template to typeset your own
paper. In this case we encourage you to also read the \LaTeX\ comments
embedded in the document.

%-------------------------------------------------------------------------
\section{Instructions}

Please read the following carefully.

%-------------------------------------------------------------------------
\subsection{Language}

All manuscripts must be in English.

%-------------------------------------------------------------------------
\subsection{Margins and page numbering}

All printed material, including text, illustrations, and charts,
must be kept within a print area %6.31 inches (16.03 cm) 
7 inches (17.7 cm) wide by
%9.10 inches (23.13 cm) 
9.44 inches (24 cm) high. Do not write or print anything
outside the print area. Number your pages on odd sites right
above, on even sites left above, no page number on the first site.
Do not use page numbering within the final version of your paper.


%------------------------------------------------------------------------
\subsection{Formatting your paper}

% All text with the exception of the abstract must be in a two-column format.
% The total allowable width of the text area -- including header and footer
% lines -- is 161\,mm (6.34 inch) wide by 231\,mm (9.10 inch) high.
% 
% Columns are to be 76\,mm (3.0 inch) wide, with a 8\,mm (0.315 inch) space 
% between them.

All text with the exception of the abstract must be in a two-column format.
The total allowable width of the text area -- including header and footer
lines -- is 177\,mm (7 inch) wide by 245\,mm (9.64 inch) high.

Columns are to be 84\,mm (3.3 inch) wide, with a 8\,mm (0.315 inch) space 
between them.

The space between the header line and the first line of the text body and
between the last line of the text body and the footer line is 5\,mm
(0.196 inch) each.

%-------------------------------------------------------------------------
\subsection{Type-style and fonts}


The ABSTRACT ist to be in a one-column format. The MAIN TEXT is to be in a
two-column format.

MAIN TEXT. Type main text in 9-point Times, single-spaced. Do \emph{not} use
double-spacing. All paragraphs should be indented 1 em (the length of the
dash in the actual font). Make sure your text is fully justified -- that is,
flush left and flush right. Please do not place any additional blank lines
between paragraphs. Figure and table captions should be 9-point Times
boldface type as in Figure~\ref{fig:firstExample}.

\noindent Long captions should be set as in Figure~\ref{fig:ex1} or
Figure~\ref{fig:ex3}.

\begin{figure}[htb]
   % an empty figure just consisting of the caption lines
   \caption{\label{fig:ex1}
     'Empty' figure only to serve as an example of long caption requiring 
     more than one line. It is not typed centered but aligned on both sides.}
\end{figure}

\noindent
Figures which need the full textwidth can be typeset as Figure~\ref{fig:ex3}.

%-------------------------------------------------------------------------
\subsection{Footnotes}

Please do \emph{not} use footnotes at all!\footnote{I mean it}\footnote{Except for refering to web pages}


%-------------------------------------------------------------------------
\subsection{References}

List all bibliographical references in 9-point Times, single-spaced, at the
end of your paper in alphabetical order. When referenced in the text, enclose
the citation index in square brackets, for example~\cite{Lous90}. Where
appropriate, include the name(s) of editors of referenced books.

For your references please use the following algorithm:
\begin{itemize} 
\item \textbf{one} author: first 3 chars plus year -- 
      e.g.\ \cite{Lous90}
\item \textbf{two}, \textbf{three} or \textbf{four} authors: first char
      of each family name plus year --  e.g.\ \cite{Fellner-Helmberg93} 
      or \cite{Kobbelt97-USHDR} or \cite{Lafortune97-NARF}
\item \textbf{more than 4} authors: first char of family name from 
      first 3 authors followed by a '*' followed by the year -- 
      e.g.\ \cite{Buhmann:1998:DCQ} or \cite{FolDamFeiHug.etal93} 
\end{itemize}

For BibTeX users a style file \ \texttt{eg-alpha.bst} and 
\texttt{eg-alpha-doi.bst} \ is available which uses the above algorithm.

For Biber users a style file \ \texttt{EG.bbx} \ is available which uses the above algorithm.

%-------------------------------------------------------------------------
\subsection{Illustrations, graphs, and photographs}

All graphics should be centered.

%%%
%%% Figure 1
%%%
\begin{figure}[htb]
  \centering
  % the following command controls the width of the embedded file
  % (relative to the width of the current column)
  \includegraphics[width=.8\linewidth]{oth.png}
  \parbox[t]{.9\columnwidth}{\relax
           For all figures please keep in mind that you \textbf{must not}
           use images with transparent background! 
           }
  %
  \caption{\label{fig:firstExample}
           Here is a sample figure.}
\end{figure}


%-------------------------------------------------------------------------
\subsection{Color}

\textbf{Please observe:} as of 2003 publications in the proceedings of the
Eurographics Conference can use color images throughout the paper. No
separate color tables are necessary.

However, workshop proceedings might have different agreements! 
Figure~\ref{fig:ex3} is an example for creating color plates.

%------------------------------------------------------------------------
\subsection{Embedding of Hyperlinks / Typesetting of URLs}

Due to the use of the package \texttt{hyperref} the original behavior
of the command $\backslash$\texttt{url} from the package \texttt{url}
is not available. To circumvent this problem we either recommend to
use the command $\backslash$\texttt{httpAddr} from the 
included package \texttt{egweblnk} (see below) or to replace the
command $\backslash$\texttt{url} by the command $\backslash$\texttt{webLink} 
-- e.g. in cases where $\backslash$\texttt{url} has been used
widely in BibTeX-References. In the latter case we suggest to run
BibTeX as usual and then replace all occurences of $\backslash$\texttt{url}  by
$\backslash$\texttt{webLink}

\noindent
The provided commands for hyperlinks, in a nutshell, are:

\begin{description} \itemsep 1ex
\item [\webLinkFont $\backslash$httpAddr \{URL without leading 'http:'\}]
      \mbox{}\\
      e.g. \  \httpAddr{//diglib.eg.org/handle/10.2312/306}

\item [\webLinkFont $\backslash$httpsAddr \{URL without leading 'https:'\}]
      \mbox{}\\
      e.g. \  \httpsAddr{//diglib.eg.org/handle/10.2312/306}      

\item[\webLinkFont $\backslash$ftpAddr \{URL without leading 'ftp:'\}]
      \mbox{}\\
      e.g. \  \ftpAddr{//www.eg.org/EG/DL/ftpupload}   %

\item[\webLinkFont $\backslash$URL \{url\}]
      \mbox{}\\
      e.g. \  \URL{http://diglib.eg.org/handle/10.2312/306}

\item[\webLinkFont $\backslash$MailTo \{Email addr\}]
      \mbox{}\\
      e.g. \  \MailTo{publishing@eg.org}

\item[\webLinkFont $\backslash$MailToNA \{emailName\}\{@emailSiteAddress\}]
      \mbox{}\\
      e.g. \  \MailToNA{publishing}{@eg.org}

\item[\webLinkFont $\backslash$webLink\{URL without hyperlink creation\}]
      \mbox{}\\
      e.g. \  \webLink{http://www.eg.org/some_arbitrary_long/but_useless/URL}

\end{description}


%------------------------------------------------------------------------
\subsection{PDF Generation}

Your final paper should be delivered as a PDF document with all typefaces
embedded. \LaTeX{} users should use \texttt{dvips} and \texttt{ps2pdf} to
create this PDF document. Adobe Acrobat Distiller may be used in place of
\texttt{ps2pdf}.

Adobe PDFWriter is \emph{not} acceptable for use. Documents created with
PDFWriter will be returned to the author for revision. \texttt{pdftex} and
\texttt{pdflatex} (and its variants) can be used only if the author can
make certain that all typefaces are embedded and images are not downsampled
or subsampled during the PDF creation process.
FOR WIS PLEASE USE PDFLATEX.

The PDF file \emph{must not} be change protected.

%------------------------------------------------------------------------
\subsubsection*{Configuration Notes: dvips / ps2pdf / etc.}

\noindent
\texttt{dvips} should be invoked with the \texttt{-Ppdf} and \texttt{-G0}
flags in order to use Type 1 PostScript typefaces:

\begin{verbatim}
    dvips -t a4 -Ppdf -G0 -o my.ps my.dvi
\end{verbatim}


\noindent
If you are using version 7.x of GhostScript, please use the following method of invoking \texttt{ps2pdf}, in
order to embed all typefaces and ensure that images are not downsampled or subsampled in the PDF
creation process:

\begin{verbatim}
  ps2pdf -dMaxSubsetPct=100 \
         -dCompatibilityLevel=1.3 \
         -dSubsetFonts=true \
         -dEmbedAllFonts=true \
         -dAutoFilterColorImages=false \
         -dAutoFilterGrayImages=false \
         -dColorImageFilter=/FlateEncode \
         -dGrayImageFilter=/FlateEncode \
         -dMonoImageFilter=/FlateEncode \
         mypaper.ps mypaper.pdf
\end{verbatim}


If you are using version 8.x of GhostScript, please use this method in place of the example above:
\begin{verbatim}
  ps2pdf -dPDFSETTINGS=/prepress \
         -dCompatibilityLevel=1.3 \
         -dAutoFilterColorImages=false \
         -dAutoFilterGrayImages=false \
         -dColorImageFilter=/FlateEncode \
         -dGrayImageFilter=/FlateEncode \
         -dMonoImageFilter=/FlateEncode \
         -dDownsampleColorImages=false \
         -dDownsampleGrayImages=false \
         mypaper.ps mypaper.pdf
\end{verbatim}

%------------------------------------------------------------------------
\subsubsection*{Configuration Notes: pdftex / pdflatex / etc.}

\noindent
Configuration of these tools to embed all typefaces can be accomplished by editing the \texttt{updmap.cfg} file
to enable inclusion of the standard (or base) 14 typefaces.

Linux users can run the \texttt{updmap} script to do this:
\begin{verbatim}
updmap --setoption pdftexDownloadBase14 true
\end{verbatim}

Windows users should edit the \texttt{updmap.cfg} files found in their TeX installation directories (one or both
of the following may be present):
\begin{verbatim}
  INSTALLDIR\texmf\web2c\updmap.cfg
  INSTALLDIR\localtexmf\miktex\config\updmap.cfg
\end{verbatim}

Ensure the value for \texttt{pdftexDownloadBase14} is "true," and then follow the instructions found here:
\httpAddr{//docs.miktex.org/manual/} to update your MikTeX installation.

%------------------------------------------------------------------------
\subsection{Exclusive License Form}

You must include your signed Eurographics Exclusive License Form 
when you submit your finished paper. We MUST have this form before
your paper can be published in the proceedings.

%-------------------------------------------------------------------------
\subsection{Conclusions}

Please direct any questions to the production editor in charge of
these proceedings.

%-------------------------------------------------------------------------
% bibtex
\bibliographystyle{eg-alpha-doi} 
\bibliography{bib}       

% biblatex with biber
% \printbibliography                

%-------------------------------------------------------------------------
\newpage


\begin{figure*}[tbp]
  \centering
  \mbox{} \hfill
  % the following command controls the width of the embedded PS file
  % (relative to the width of the current column)
  \includegraphics[width=.3\linewidth]{oth.png}
  % replacing the above command with the one below will explicitly set
  % the bounding box of the PS figure to the rectangle (xl,yl),(xh,yh).
  % It will also prevent LaTeX from reading the PS file to determine
  % the bounding box (i.e., it will speed up the compilation process)
  % \includegraphics[width=.3\linewidth, bb=39 696 126 756]{sampleFig}
  \hfill
  \includegraphics[width=.3\linewidth]{oth.png}
  \hfill \mbox{}
  \caption{\label{fig:ex3}%
           For publications with color tables (i.e., publications not offering
           color throughout the paper) please \textbf{observe}: 
           for the printed version -- and ONLY for the printed
           version -- color figures have to be placed in the last page.
           \newline
           For the electronic version, which will be converted to PDF before
           making it available electronically, the color images should be
           embedded within the document. Optionally, other multimedia
           material may be attached to the electronic version. }
\end{figure*}

